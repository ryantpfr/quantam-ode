\documentclass[10pt]{article}
\usepackage[margin=1.0in]{geometry} % Please keep the margins at 1.5 so that there is space for grader comments.
\usepackage{amsmath,amsthm,amssymb,tikz}
\def\mathLarge#1{\mbox{\LARGE $#1$}}

\newcommand{\overbar}[1]{\mkern 1.5mu\overline{\mkern-1.5mu#1\mkern-1.5mu}\mkern 1.5mu}

\begin{document}
\begin{center}\Large Math 492 Quantam Physics Problem\end{center}
\begin{center}\large Ryan Toepfer and Carter Ronald\end{center}

\section{Problem Statement}
We have set out to analyze how the parameters of quantam systems govern how the state of the system changes over time.

\section{Axioms and Definitions}

Definition 2.1) A complex $A$ matrix is \textbf{skew hermitian} iff its transpose conjugate is equal to $-A$.
Therefore, all $2\times2$ skew hermitian matricies fit the form
\begin{center}
$\begin{pmatrix}ai&b+ci\\-b+ci&di\end{pmatrix}$
or alternatively, 
$\begin{pmatrix}ai&\beta\\-\bar\beta&di\end{pmatrix}$
\end{center}

Axiom 2.2) The state of a quantam system can be described entirely by $\begin{pmatrix}x\\y\end{pmatrix}$, where $x,y$ are complex numbers. 

Axiom 2.3) Two quantam states are considered the same if one is a multiple of another, i.e. $\begin{pmatrix}x\\y\end{pmatrix}=\alpha\begin{pmatrix}v\\w\end{pmatrix}$ for any $x,y,\alpha,v,w\in\mathbb C$, means both states are the same.
Therefore any state $\begin{pmatrix}x\\y\end{pmatrix}$ can be represented as a ratio $z=x/y$.

Axiom 2.4) The function describing the state of a quantam system at time $t$ is the solution to the ODE
\begin{center}$
\dot{\begin{pmatrix}x\\y\end{pmatrix}}=A\begin{pmatrix}x\\y\end{pmatrix}
$\end{center}
where $A$ is a skew hermation matrix.

\section{Initial Results}
Combining Axiom 2.4 with the general form of a $2\times2$ skew hermation matrix gives
\begin{center}
$\dot{\begin{pmatrix}x\\y\end{pmatrix}}=\begin{pmatrix}ai&\beta\\-\bar\beta&di\end{pmatrix}\begin{pmatrix}x\\y\end{pmatrix}$
or $\begin{cases}\dot x=aix+\beta y\\\dot y=-\bar\beta x+diy\end{cases}$
\end{center}
So if you consider $z=y/x$, and its derivative, then we get the following new equation
\begin{align*}
\dot z&=\frac{\dot yx-y\dot x}{x^2}\\
\dot z&=\frac{(-\bar\beta x+diy)x-y(aix+\beta y)}{x^2}\\
\dot z&=\frac{-\bar\beta x^2+diyx-aixy-\beta y^2}{x^2}\\
\dot z&=-\bar\beta+diz-aiz-\beta z^2
\end{align*}
From here we can make assumptions to make this easier to solve

\section{Assume $\beta=0, d=-a, a\neq0$}

Adding these assumptions to our initial result gives:
\begin{align*}
\dot z&=-2aiz
\end{align*}
Now we can define $f,g$ as the real and complex parts of $z$ respectively
\begin{align*}
\dot f+i\dot g&=-2ai(f+ig)\\
\dot f+i\dot g&=-2aif+2ag\\
\end{align*}
Giving the system:
\begin{align*}
\begin{cases}\dot f=2ag\\\dot g=-2af\end{cases}\\
\therefore \ddot f=2a\dot g\\
\ddot f=-4a^2f\\
\ddot f+4a^2f=0\\
\lambda^2+4a^2=0\\
\lambda^2=-4a^2\\
\lambda = \pm\sqrt{-4a^2} = \pm2ai\\
\end{align*}
So the solution is
\begin{align*}
f&=c_1e^0\cos(2at)+c_2e^0\sin(2at)\\
f&=c_1\cos(2at)+c_2\sin(2at)\\
\therefore \dot f&=-c_12a\sin(2at)+c_22a\cos(2at)
\end{align*}
plugging $f$ into our equation for $\dot f$ gives $g$: 
\begin{align*}
\dot f&=2ag\\
-c_12a\sin(2at)+c_22a\cos(2at)&=2ag\\
-c_1\sin(2at)+c_2\cos(2at)&=g\\
\end{align*}
As a vector gives:
\begin{align*}
\begin{pmatrix}f\\g\end{pmatrix}=
\begin{pmatrix}c_1\cos(2at)+c_2\sin(2at)\\-c_1\sin(2at)+c_2\cos(2at)\end{pmatrix}
\end{align*}

\includegraphics[width=\textwidth]{Figure1}
Changing $c_1$ and $c_2$ affects the size of the circle.
Changing $a$ only affects the paramaterization.

Now let's begin an analysis of the radius of the circle.
Notice that $z\bar z=(f+g)(f-gi)=f^2+g^2$, which is the radius of our circle. 
Since this is a constant with respect to time, we should expect $\frac{d}{dt}\left(z\bar z\right)=0$.
Let's check this assumption:
\begin{align*}
\frac{d}{dt}\left(z\bar z\right)&=\dot z\bar z+z\dot {\bar z}\\
&=(-2aiz)\bar z+z\overbar{(-2aiz)}\\
&=(-2aiz)\bar z+z\overbar{(-2ai)}\bar z\\
&=(-2aiz)\bar z+z(2ai)\bar z\\
&=0
\end{align*}
So if the radius of the circle never changes it is just equal to the initial value $z(0)\bar z(0)$.

\section{Assuming $\beta$ is real}

Let $\beta=b\in\mathbb R$.
Recall our ODE.
$$\dot{\begin{pmatrix}x\\y\end{pmatrix}}=A{\begin{pmatrix}x\\y\end{pmatrix}}, A=\begin{pmatrix}0&\beta\\-\beta&0\end{pmatrix}$$
Here we are going to take a different approach than what we did in section 4. As we saw when $\beta=0$ the matrix:
\begin{align*}
A = \begin{pmatrix}ai&0\\0&-ai\end{pmatrix}
\end{align*}
results in a situation where $z = constant$ since $\dot z = 0$. For our current assumptions we have:
\begin{align*}
A = \begin{pmatrix}0&b\\-b&0\end{pmatrix}
\end{align*}
and it follows that we could diagonalize this matrix to get the same situation as the previous case where $b = 0$. To diagonalize the matrix above we need to find a matrix $T$ such that:
\begin{align*}
T\begin{pmatrix}0&b\\-b&0\end{pmatrix}T^{-1}
\end{align*}
is diagonal. Furthermore we need to modify the original system of equations we started with.
\begin{align*}
\dot{\begin{pmatrix}x\\y\end{pmatrix}}&=\begin{pmatrix}0&b\\-b&0\end{pmatrix}\begin{pmatrix}x\\y\end{pmatrix}\\
T\dot{\begin{pmatrix}x\\y\end{pmatrix}}&=T\begin{pmatrix}0&b\\-b&0\end{pmatrix}\begin{pmatrix}x\\y\end{pmatrix}\\
T\dot{\begin{pmatrix}x\\y\end{pmatrix}}&=T\begin{pmatrix}0&b\\-b&0\end{pmatrix}T^{-1}T\begin{pmatrix}x\\y\end{pmatrix}
\end{align*}
Now we have the original system in terms of the diagonal matrix that we want. To simplify, let: 
\begin{center}
$\begin{pmatrix}u\\v\end{pmatrix}=T\begin{pmatrix}x\\y\end{pmatrix}$
\end{center}
Then
\begin{center}
$u = T_{1,1}x + T_{1,2}y$\\
$v = T_{2,1}x + T_{2,2}y$
\end{center}
To find T we need to find the eigenvalues and their corresponding eigenvectors for the matrix:
\begin{align*}
A = \begin{pmatrix}0&b\\-b&0\end{pmatrix}
\end{align*}
Which are
\begin{center}
$\lambda_{1} = ib$, $v_{1} = \begin{pmatrix}1\\i\end{pmatrix}$\\
$\lambda_{2} = -ib$, $v_{2} = \begin{pmatrix}i\\1\end{pmatrix}$\\
\end{center}
Thus
\begin{align*}
T &= \begin{pmatrix}1&i\\i&1\end{pmatrix}\\
T^{-1} &= \begin{pmatrix}\frac{1}{2}&\frac{-i}{2}\\[6pt]\frac{-i}{2}&\frac{1}{2}\end{pmatrix}\\
\end{align*}
Plugging the values of T into the previous system involving $u$ and $v$ we get
\begin{center}
$u = 1x + iy$\\
$v = ix + 1y$
\end{center}
Now we want to consider $|vu^{-1}|$ similarly to when we considered $z=yx^{-1}$ which gives us
\begin{center}
$vu^{-1} = \mathLarge{\frac{ix + y}{x + iy}}$
\end{center}
We can write this in terms of $z$ with some manipulation as follows
\begin{center}
$vu^{-1} = \mathLarge{\frac{ix + y}{x + iy}}\cdot\mathLarge{\frac{x^{-1}}{x^{-1}}}=
\mathLarge{\frac{i + yx^{-1}}{1 + iyx^{-1}}} = \mathLarge{\frac{i + z}{1 + iz}}$
\end{center}
Therefore
\begin{center}
$|vu^{-1}| = \mathLarge{\frac{|i + z|}{|1 + iz|}} = c$
\end{center}
where $c \in \mathbb C$ is a constant. Solving this for $z$ results in the following


1:
$$z = \frac{c-i}{1-ci}=\frac{(c-i)(1+ci)}{1+c^2}=\frac{-i+c^2i+2c}{1+c^2}$$
2:
$$ z = \frac{-c-i}{1+ci}=\frac{-(c+i)(1-ci)}{{1+c^2}}=\frac{-i+c^2i-2c}{{1+c^2}}$$

So in General
$$ z = \frac{-i+c^2i\pm2c}{{1+c^2}}$$

\section{Assuming $\beta$ is real, a=d=0}
This gives us the equation
\begin{align*}
\dot z&=-\bar\beta+diz-aiz-\beta z^2\\
&=-\bar\beta-\beta z^2\\
&=-\beta(1+z^2)
\end{align*}
Which is separable and has the solution $z=\tan(-\beta t+C)$



\end{document}